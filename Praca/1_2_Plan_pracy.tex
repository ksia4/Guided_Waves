\section{Plan pracy}
\label{sec:plan_pracy}

Praca podzielona jest na 6 rozdziałów. W rozdziale 1 przedstawiono krótki wstęp oraz założenia projektowe. W rozdziale 2 opisane są podstawowe pojęcia dotyczące propagacji fali w środowisku sprężystym. Podano definicję naprężenia oraz odkształcenia oraz opisano rodzaje fal sprężystych. Następnie przytoczone są informację o zjawisku dyspersji. Opisano przykład kiedy możliwe jest analityczne wyznaczenie krzywych dyspersji oraz przykład gdzie nie jest to już możliwe. Na kolejnych stronach znajdują się metody numerycznego wyznaczania tych krzywych oraz możliwości wyznaczenia ich w badaniach doświadczalnych. Dwie ostatnie sekcje rozdziału skupione są na możliwościach wyznaczania krzywych wzbudzalności oraz ekefcie sing-around. Rozdział 3 przedstawia zagadnienia metody elementów skończonych, które zostały wykorzystane w aplikacji do obliczania modelu pręta. Opisane są kolejno funkcje kształtu, sposoby ich wyznaczania, obliczanie macierzy mas i sztywności elementu oraz agregacja macierzy globalnych. Dla wyznaczonego modelu przedstawiono kilka sposobów wyznaczenia rozwiązania w postaci przemieszczeń, oraz sposoby estymacji błędów rozwiązania i sprawdzenia czy rozwiązanie jest zbieżne. W rozdziale 4 przedstawiono trzy wybrane metody kompensacji dyspersji, które zostały zaimplementowane w programie. Każda metoda opisana została zarówno od strony teoretycznej jak i praktycznej implementacji numerycznej. Ostatni podrzodział stanowi podsumowanie, porównujące wszystkie omawiane metody. Opisane są tam wady i zalety każdej z nich oraz ich ograniczenia. Rozdział 5 stanowi opis zbudowanej aplikacji oraz opis środowiska programistycznego. Początek rozdziału zawiera instrukcję instalacji i konfiguracji interpretera Python programu PyCharm  który został wykorzystany w projekcie. Znajduje się tam rownież lista wszystkich bibliotek niezbędnych do uruchomienia stworzonej aplikacji. Dalsza część rozdziału zawiera pełny opis działania programu wraz z przytoczonymi przykładami wykorzystania. Opisane są kolejne kroki obliczeń dwóch głównych modułów aplikacji tj. budowania modelu wraz z wyznaczaniem krzywych dyspersji oraz kompensowania dyspersji z użyciem wybranej metody.Dodatkowymi elementami jest możliwość symulowania efektu sing-around z pomocą krzywych wzbudzalności oraz korzystanie z zaimplementowanego graficznego interfejsu użytkownika. Rozdział 6 jest podsumowaniem efektów pracy przy projekcie. Opisano stan zrealizowania założeń oraz podano możliwości rozwoju projektu.




















