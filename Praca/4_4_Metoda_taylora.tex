
\section{Metoda mapowania liniowego przy pomocy rozwinięcia w szereg Taylora}
\label{sec:Taylor}
Prezentowana w poniższej sekcji metoda kompensacji dyspersji została przedstawiona w artykule []\textcolor{red}{referenacja do yuanliu}
\subsection{Podstawy teoretyczne}
Po wzbudzeniu badanego pręta odpowiednim sygnałem wejściowym, możliwe jest aby nadajnik przełączył się w tryb obioru sygnału i nasłuchiwał nadejścia odpowiedzi układu. W takim wypadku odpowiedź jaką uzyskamy będzie sumą odpowiedzi ze wszystkich odbić, od końca pręta oraz ewentualnych łączeń lub uszkodzeń.  Proponowana w tym rozdziale metoda opiera się na założeniu, że w badanym obiekce propaguje jedna wybrana postać drgań. Jej celem jest kompensacja powstałej dyspersji, tak aby sygnały z różnych punktów odbicia nie nachodziły na siebie i była możliwa ich interpretacja w celu ustalenia ilości punktów odbicia oraz oszacowania ich odegłości od miejsca wzbudzenia na podstawie znajomości prędkości grupowej fali. Przy tak sformułowanych założeniach, sygnał otrzymany w odbiorniku można przedstawić wzorem:
\begin{equation}
g(t) = \sum\limits_{n=1}{N}f_n(r_n,t)=\frac{1}{2\pi}\int _{-\infty}^{\infty}F(\omega)\sum\limits_{n=1}^{N}(A_n(\omega)e^{-ikr_n})e^{i\omega t} d\omega \label{eq:g(t)_taylor}
\end{equation}
Gdzie:

$N$ - całkowita liczba ścieżek propagacji sygnału (liczba punktów odbicia)

$r_n$ - długość n-tej ścieżki propagacji

$A_n$ - współczynnik odbicia n-tego punktu odbicia.

Widmo częstotliwości takiego sygnału $G(\omega)$  możemy obliczyć przy pomocy transformaty Fouriera i zapisać wzorem:
\begin{equation}
G(\omega) = F(\omega)\sum\limits_{n=1}^{N}(A_n(\omega)e^{-ikr_n} \label{eq:G(omega)_taylor}
\end{equation}

Jak wiadomo, dyspersja zależy od kształtu krzywej dyspersji, $k = K(\omega)$. Jeśli więc $K(\omega)$ jest funkcją kwadratową lub wyższego rzędu względem $\omega$, to $G(\omega)$ reprezentuje widmo częstotliwości rozproszonych pakietów fal. Jeśli natomiast zależność $K(\omega)$ byłoby funkcją liniową, zjawisko dyspersji nie występowałoby, a $G(\omega)$ reprezentowałoby widmo częstotliwości sygnału, który nie uległ dyspersji. 
Wybraną krzywą dyspersji można przybliżyć przy pomocy rozwinięcia w szereg Taylora:
\begin{equation}
k = K(\omega) - k_0+k_1(\omega - \omega _0) + k_2(\omega - \omega _0^2)+... \label{eq:szereg_k}
\end{equation}

Gdzie:

$k_0 = \frac{\omega _0}{c_p}$

$k_1 = \frac{dk}{d\omega}|_{\omega = \omega_0}$

$k_2 = \frac{1}{2}\frac{d^2k}{d\omega ^2}|_{\omega = \omega _0}$

$c_p$ - prędkość fazowa

Wynika z tego, iż dyspersję sygnału można usunąć usuwając jej nieliniowy składnik, poprzez zastąpienie oryginalnej zależności $K(\omega)$ liniowym przybliżeniem tej funkcji. Zastosowanie takiego zabiegu sprawi, iż sygnał w dziedzinie czasu będzie postrzegany jako skompensowany do postaci niedyspersyjnej, a prędkość grupowa obliczona z liniowego przybliżenia krzywej dyspersji może zostać wykorzystana do określenia długości ścieżek propagacji wynikających z kolejnych odbić. 

W szczególnym przypadku, w którym mamy do czynienia z pojedyńczą ścieżką propagacji ($N=1$) i odległość między nadajnikiem a punktem odbicia jest znana, można usunąć dyspersję poprzez wyeliminowanie wyrażenia kwadratowego w $K(\omega)$. Matematycznie można to zrobić przy pomocy wzoru:
\begin{equation}
 \widetilde{G}(\omega)=(F(\omega)A_1(\omega)e^{-ikr_1})e^{ik_2(\omega -\omega _0)^2r_1}
\end{equation}

Gdzie $\widetilde{G}(\omega)$ jest zmodyfikowanym widmem częstotliwości, a $r_1$ jest odległością między nadajnikiem i odbiornikiem. Sygnał skompensowany w dziedzinie czasu można natychmiast uzyskać poprzez odwrotną transformatę Fouriera. Ponieważ jednak zazwyczaj długość ścieżki propagacji sygnału nie jest znana, a sygnał może składać się z wielu odbić, między innymi od uszkodzeń ($N>1$) usuwanie dyspersji przy pomocy powyższego wzoru byłoby niepraktyczne. 

Rysunek \ref{fig:krzywa_taylorem} obrazuje przykład krzywej dyspersji, trybu $A_0$, płyty aluminiowej o grubości 3,175 mm. Linia ciągła pokazuje zależności $K(\omega)$ uzyskaną metodami komputerowymi, natomiast linia przerywana kropkowana wskazuje z rozwinięcie szeregu Taylora pierwszego rzędu, a linia przerywana rozszerzenie szeregu Taylora drugiego rzędu w odniesieniu do centralnej częsości kątowej ($\omega _0 = 2\pi *50 kHz$)

\begin{figure}[h]
\centering
\includegraphics[width=14cm]{Zdjecia/4/buba}
\caption{Przykładowe porównanie oryginalnej krzywej oraz jej przybliżeń przy pomocy rozwinięcia w szereg Taylora}
\label{fig:krzywa_taylorem}
\end{figure}

Łatwo zauważyć, że po pierwsze, rozszerzenie drugiego rzędu daje bardzo dobrze przybliżenie pierwotnego kształtu krzywej, po drugie, $K(\omega)$ jest monotoniczną funkcją $\omega$ w otoczeniu $\omega _0$. 

Równanie \ref{eq:G(omega)_taylor} można zapisać w postacji złożenia funkcji:
\begin{equation}
G(\omega) = G(k)\circ K(\omega)
\end{equation}

Gdzie $\circ$ jest operatorem składania funkcji. $G(k)$ jest niejawną funkcją k z równania \ref{eq:G(omega)_taylor}. Zamieniając $K(\omega)$ na $K_{lin}(\omega)$ będące aproksymacją $K(\omega)$ pierwszego rzędu w punkcie $\omega _0$, gdzie $\omega _0$ oznacza częstotliwość o najwyższej energii, można wyprowadzić zmodyfikowane widmo częstotliwości:
\begin{equation}
\widetilde{G}(\omega) = G(k)\circ K_{lin}(\omega)
\end{equation}

Ponieważ $G(\omega)$ jest znane oraz znana jest analizowana krzywa dyspersji, znane jest również $G(k)$, obliczając przybliżenie liniowe $K_{lin}(\omega)$ można w prosty psosób obliczyć $\widetilde{G}(\omega)$, poprzez interpolację odpowiednich wartości. Opisana metoda może być określana mianem mapowania liniowego. W jej efekcie uzyskane zostaje nowe widmo częstotliwości $\widetilde{G}(\omega)$. Po mapowaniu widmo amplitudy pozostaje bez zmian, natomiast widmo fazy stopniowo odbiega od pierwotnego w miarę oddalania się od wybranej, środkowej częstotliwości, co dobrze ilustruje rysynek \ref{fig:widma}

\begin{figure}[h]
\centering
\includegraphics[width=14cm]{Zdjecia/4/widma}
\caption{Przykładowe porównanie oryginalnych charakterystyk oraz ich przybliżeń przy pomocy rozwinięcia w szereg Taylora}
\label{fig:widma}
\end{figure}

Należy zaznaczyć, iż stosowanie omawianej metody jest możliwe tylko w sytuacji, gdy $K(\omega)$ jest funkcją monotoniczną

\subsection{Implementacja numeryczna}
Implementacja prezentowanej metody opiera się głównie na znajomości krzywej dyspersji, której propagację bierzemy pod uwagę. Pierwszym krokiem, jest wygenerowanie odpowiedniego sygnału testowego. Mając wygenerowany sygnał można przystąpić do właściwego procesu kompensacji. W pierwszej kolejności analizowane jest widmo amplitudowe otrzymanego sygnału. Na jego podstawie uzyskiwana jest informacja o częstotliwości z największą energią. Zostaje ona wybrana na częstotliwość w której nastąpi przybliżenie liniowe. Po wybraniu $\omega _0$ odnajdywane jest na krzywej dyspersji odpowiednia wartość liczby falowej. Korzystając z zależności opisujących wartości $k_0$ i $k_1$ wyliczone zostaje liniowe przybliżenie badanej krzywej. Uzyskane w aplikacji wyniki przedstawia rysunek \ref{fig:krzywa_moja}. Linia niebieska prezentuje oryginalną krzywą dyspersji, natomiast linia zielona przdstawia jej przybliżenie uzyskane w aplikacji przy użyciu rozwinięcia w szereg Taylora.

\begin{figure}[h]
\centering
\includegraphics[width=14cm]{Zdjecia/4/krzywa_moja}
\caption{Przykładowe porównanie oryginalnej krzywej oraz jej przybliżeń przy pomocy rozwinięcia w szereg Taylora}
\label{fig:krzywa_moja}
\end{figure}

Kolejnym krokiem implementowanego algorytmu, jest wyliczenie G(k) na podstawie otrzymanego widma sygnału $G(\omega)$ oraz krzywej dyspersji. Następnie ponowne wyznaczenie zależności w dziedzinie częstotliwości, tym razem jednak używając przybliżenia liniowego zamiast oryginalnej zależności $k(\omega)$. Omawiany algorytm doskonale ilustruje rysunek \ref{fig:algo_Taylora}

\begin{figure}[h]
\centering
\includegraphics[width=13cm]{Zdjecia/4/algo_Taylora}
\caption{Wizualizacja algorytmu []Przypis do Pialucha}
\label{fig:algo_Taylora}
\end{figure}

Przedstawia on krzywą $G(\omega)$ z aż trzema osiami poziomymi. Linia ciągła przedstawia zależność $G(K(\omega))$, czerwonymi kółkami oznaczony jest sygnał $G(\omega)$. Jak widać przejście z dziedziny $K(\omega)$ na $\omega$ jest nieliniowe co wynika z nieliniowego charakteru krzywej dyspersji. Zielonymi trójkątami oznaczono natomiast krzywą $G(K_{lin}(\omega))$. Przejście z dziedziny $\omega$ na $K_{lin}(\omega)$ jest przejściem o charakterze liniowym. 
\subsection{Wybrane wyniki symulacji}
Dokładność odwzorowania przybliżenia liniowego oryginalnej krzywej zależy od jej kształtu. Im bardziej liniowa jest charakterystyka w pewnym otoczeniu wybranej częstotliwości, tym mniej widoczne będzie zjawisko dyspersji i tym dokładniejsze będzie odwzorowanie. Rysunek \ref{fig:taylor porownanie} pokazuje wygenerowane przybliżenie krzywej dyspersji odpowiadającej pierwszej postaci oraz jej liniowego przybliżenia. Rysunek \ref{fig:taylor porownanie2} krzywą odpowiadającą trzeciej postaci oraz jej liniowego przybliżenia. Łatwo można zauważyć, że w przypadku pierwszej krzywej największe różnice są przy bardzo niskich częstotliwościach a dalej funkcja uzyskuje kształt niemalże liniowy. W przypadku drugiej krzywej dopasowanie jest zdecydowanie mniej dokładne

\begin{figure}[h]
\centering
\includegraphics[width=13cm]{Zdjecia/4/taylorporownanie}
\caption{Porównanie oryginalnej krzwej z jej liniowym przybliżeniem}
\label{fig:taylor porownanie}
\end{figure}

\begin{figure}[h]
\centering
\includegraphics[width=13cm]{Zdjecia/4/taylor_porownanie_trzecie}
\caption{Porównanie oryginalnej krzwej z jej liniowym przybliżeniem}
\label{fig:taylor porownanie2}
\end{figure}

Kolejny rysunek przedstawia przykład sygnału przed i po kompensacji. W prezentowanym przykładzie propagowała pierwsza postać drgań, na odległość dwóch metrów. Sygnał skompensowany w znacznie większym stopniu przypomina sygnał wejściowy. Również czas jego trwania nieco się skrócił. Ze względu na małą dyspersyjność wybranego trybu, wydłużenie w czasie w prezentowanym przykładzie nie było bardzo znaczne, niemniej jednak stosując omawianą metodą kompensacji czas trwania sygnału udało się skrócić do wartości zbliżonej do długości sygnału oryginalnego.

\begin{figure}[h]
\centering
\includegraphics[width=13cm]{Zdjecia/4/taylor_pierwszy_prim}
\caption{Porównanie sygnału przed i po kompensacji}
\label{fig:taylor pierwsza}
\end{figure}


Na ostatnim rysunku \ref{fig:przedipo2} przedstawione zostało porównanie otrzymanych w wyniku kompensacji sygnałów z sygnałem zadanym. Sygnał nie został odzyskany do postaci idealnie odpowiadającej pierwotnemu sygnałowi, co wynika bezpośrednio z zastosowanej procedury, jednak cel kompensacji, jakim jest skrócenie czasu trwania otrzymanego sygnału został osiągnięty. Dodatkowym atutem prezentowanej metody w stosunku do tej omówionej w poprzednim podrzodziale jest brak konieczności znajomości długości ścieżki propagacji. Pozwala ona na badanie obiektu bez zalożeń o odleglości na jaką sygnał będzie propagował oraz ilości jego odbić w obiekcie.
\begin{figure}[h]
\centering
\includegraphics[width=13cm]{Zdjecia/4/przedipo2}
\caption{Porównanie sygnału oryginalnego z sygnałem skompensowanym}
\label{fig:przedipo2}
\end{figure}
