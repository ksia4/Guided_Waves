\chapter{Podsumowanie}
\label{cha:podsumowanie}

Celem ninijszej pracy było stworzenie aplikacji, pozwalającej na symulację fali prowadzonej w długim stalowym pręcie o zadanych parametrach. Stworzenie opisywanej aplikacji wymagało dokładnego zrozumienia poruszanych zagadnień. Niezbędne było pogłębienie wiedzy w dziedznie liniowej teorii sprężystości, zapoznanie się z rodzajami fal sprężystych, zrozumienie ich natury oraz zrozumienie zjawiska dyspersji. Kolejnym etapem było zapoznanie się z zagadnieniem elementów skończonych. Pierwszym krokiem w realizacji było opracowanie oprogramowania wyliczającego krzywe dyspersji na podstawie zadanych parametrów. Aby to osiągnąć, niezbędne było zapoznanie się z zasadami budowy funkcji kształtu. Następnie niezbędne było opracowanie algorytmów wyznaczających macierze mas oraz sztywności pojedynczych elementów skończonych. Kolejnym etapem było opracowanie algorytmów agregacji globalnych macierzy mas i sztywności. Zadaniem kolejnego algorytmu było rozwiązanie wyznaczonego równanie macierzowego. Zaimplementowane algorytmy stanowią slover opierający swoje działanie na metodzie elementów skończonych. Stworzona aplikacje daje użytkownikowi możliwość własnoręcznego ustawienia niemal każdego parametru. Do dyspozycji są dwa rodzaje kształtu elementu skończonego: elementy czworościenne oraz sześciościenne. Istnieje również możliwość wczytania wartości z zewnętrznego programu o nazwie MARC. Wygenerowane krzywe przechowują informacje niezbędne do symulacji propagacji fali prowadzonej. 

Kolejnym etapem było opracowanie algorytmu do symulacji propagacji fali. Aby było to możliwe zaimplementowany został algorytm agregacji danych wygenerowanych przez solver. W drugiej części pracy omówione zostały wybrane metody kompensacji dyspersji, które również zostały zaimplementowane i stanowią część prezentowanej aplikacji. Każda z proponowanych metod ma zarówno wady jak i zalety. Ważną część pracy stanowi zestawienie wyników uzyskanych z symulacji przy pomocy niniejszej aplikacji. Każda z metod została opisana od strony teoretycznej. Opisana została również ich implementacja numeryczna. Rozdział piąty poświęcony jest programistycznej stronie pracy. Opisana została tam instalacja niezbędnych narzędzi, umożliwiających sprawne korzystanie z opracowanego rozwiązania. Rozdział ten zawiera również listing ważniejszych części kodu, opis wszystkich funkcji, niezbędnych do korzystania z aplikacji wraz z przykładami użycia oraz instrukcje pozwalającą na korzystanie ze stworzonego graficznego interfejsu użytkownika. Interfejs ten stanowi zwieńczenie aplikacji. Umożliwia korzystanie ze stworzonej aplikacji nawet osobom nie znających języka programowania python. Stworzona w ramach pracy aplikacja jest kompleksowa oraz wyposażona w wiele potrzebnych funkcji. Zarówno aplikacja jak i graficzny interfejs pozostawiona jest w postaci otwartego kodu. 

Istnieje wiele możliwości dalszego rozwoju projektu. Z pewnością możliwa jest optymalizacja zastosowanych algorytmów, tak aby czas obliczeń uległ skróceniu. Dalszym krokiem rozwoju tego projektu z pewnością byłoby przetestowanie wyników symulacji na realnych obiektach, oraz weryfikacja oczekiwanych wyników z tymi uzyskanymi drogą symulacji. Kolejną możliwością rozwoju jest usprawnienie działania graficznego interfejsu oraz rozszerzenie symulacji o uwzględnienie tłumienia sygnału oraz charakterystyk wzbudzalności. 

Podsumowując, w ramach niniejszej pracy w pełni zostały zrealizowane postawione na początku cele i założenia. Aplikacja działa w pełni sprawnie, jednak pozostaje ogromny obszar możliwyego rozszerzenia powstałej pracy.