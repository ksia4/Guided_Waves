\subsection{Moduł main}
\label{cha:main}

Przykładowy skrypt wyznaczający krzywe dyspersji, z pomocą wcześniej opisanych modułów, znajduje się w module \textbf{main} głównego katalogu projektu. Kod przedstawiony jest poniżej.

\vspace{3mm}
import sympy as sp

import numpy as np

from MES\_dir import MES, config, dispersion\_curves

from MARC import functions

\vspace{3mm}
\# begin MES

x, y, z = sp.symbols('x, y, z')

if \_\_name\_\_ == "\_\_main\_\_":

\vspace{3mm}
    print("Wpisz wartość: ")

    print("1 - rysowanie krzywy dyspersji z wykorzystaniem MES")

    print("2 - rysowanie krzywych dyspersji z ostatnio policzonych danych")

    text = input()

\vspace{3mm}
    if text == '1':

\vspace{3mm}
        print("Wpisz wartość: ")

        print("4 - elementy czworościenne")

        print("8 - elemnty sześcienne")

        print("M - wczytanie macierzy z MARC i wykreślenie krzywych")

        text1 = input()

\vspace{3mm}
        \# wektor liczby falowej

        config.kvect\_min = 1e-10

        config.kvect\_max = np.pi / 4

        config.kvect\_no\_of\_points = 51

\vspace{3mm}
        \# rysowanie wykresow

\vspace{3mm}
        config.show\_plane = True

        config.show\_bar = True

        config.show\_elements = True

\vspace{3mm}

        \# zapisywanie wektorow wlasnych
        saveEigVectors = True

\vspace{3mm}
        \# obliczenia

        if text1 == '4':

            \# parametry preta

            radius = 10

            num\_of\_circles = 4

            num\_of\_points\_at\_c1 = 4

            MES.mes4(radius, num\_of\_circles, num\_of\_points\_at\_c1)

\vspace{3mm}

        if text1 == '8':

            radius = 10

            numberOfPlanes = 3

            firstCircle = 16 \#for brickMesh should be 16

            addNodes = 0

            circles = 1

            MES.mes8(numberOfPlanes, radius, circles, firstCircle, addNodes)

\vspace{3mm}
        if text1 == 'M':

            config.k, config.m = functions.getStiffAndMassMatrix()

        dispersion\_curves.drawDispercionCurves()

        print("koniec")

\vspace{3mm}
    \# rysowanie krzywych dyspersji z wczesniej obliczonych wartosci

    if text == '2':

        dispersion\_curves.drawDispercionCurvesFromFile()

\vspace{3mm}
Pierwsze linijki zapewniają dostęp do funkcji z potrzebnych modułów programu oraz bibliotek Python-a. Następnie znajduje się definicja zmiennych symbolicznych, które są wykorzystywane w programie oraz zmienne określające czy wyświetlać wykresy rozmieszczenia węzłów siatki, elementów skończonych, a także czy zapisywać do plików wektory własne. Ostatnia z tych operacji jest czasochłonna i domyślnie jest wyłączona. W dalszej części następuje seria warunków, które pozwalają na wybór obliczania nowych wartości własnych (text=1), bądź skorzystania z wcześniej wyznaczonych, do wykreślenia krzywych dyspersji (text=2). Wyboru dokonujemy poprzez wpisane odpowiedniej cyfry w konsoli programu. 

Jeśli chcemy obliczyć nowe krzywe, to będziemy mogli skorzystać z budowania modelu MES przy pomocy elementów czworościennych (text1=4), elementów sześciennych (text1=8), bądź skorzystać z modelu wyznaczonego w programie MARC (text1=M) i zapisanego w postaci plików z roszerzeniem dat w katalogu \textit{MARC}. Niezależnie od wybranego sposobu dostarczania modelu, program w kolejnej fazie przejdzie do wyznaczania krzywych dyspersji.

Siatkę dla elementów skończonych możemy dostosować przez zmianę odpowiednich wartości przyjmowanych jako argumenty funkcji \textit{mes4} oraz \textit{mes8}. Dla zmiany rodzaju siatki należy wybrać inną funkcję do jej generacji w funkcji \textit{mes4} lub \textit{mes8}. Należy przy tym pamiętać żeby dostosować także funkcję budującą elementy skończone. Przykładowo dla siatki tworzonej przy pomocy \textit{brickMesh}, powinna to być funkcja \textit{createBrickElements}.
